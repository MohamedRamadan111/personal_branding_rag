\documentclass[11pt]{article}
\usepackage[a4paper,margin=1in]{geometry}
\usepackage{amsmath,amssymb}
\usepackage{graphicx}
\usepackage{hyperref}
\usepackage{booktabs}
\usepackage{enumitem}
\usepackage{float}
\usepackage{xcolor}

\title{Efficient Attention Mechanisms: \\ BigBird and FlashAttention}
\author{Mohamed Ramadan}
\date{}

\begin{document}

\maketitle

\section{Introduction}

Transformer models rely on the self-attention mechanism, where each token attends to every other token in the sequence.

Given a sequence of length $n$, full self-attention computes:

\[
\text{Attention}(Q,K,V) = \text{Softmax}\left(\frac{QK^T}{\sqrt{d}}\right)V
\]

This requires computing an $n \times n$ attention matrix.

\textbf{Problem:}
\begin{itemize}
    \item Time Complexity: $\mathcal{O}(n^2)$
    \item Memory Complexity: $\mathcal{O}(n^2)$
\end{itemize}

This becomes infeasible for long sequences (e.g., documents, DNA, video frames).

\section{Full Attention Bottleneck}

In full attention, every token interacts with every other token.

\begin{figure}[H]
    \centering
    \includegraphics[width=\textwidth]{full - Attention.png}
    \caption{Full Self-Attention: Each token attends to all tokens ($\mathcal{O}(n^2)$).}
\end{figure}

For example, with 10 tokens:
\[
10 \times 10 = 100 \text{ attention scores}
\]

For 10,000 tokens:
\[
10^4 \times 10^4 = 10^8
\]

Clearly quadratic growth is the bottleneck.

\section{BigBird: Sparse Attention for Linear Scaling}

BigBird introduces structured sparse attention to reduce complexity from $\mathcal{O}(n^2)$ to $\mathcal{O}(n)$.

Instead of attending to all tokens, each token attends to:

\begin{itemize}
    \item \textbf{Global tokens}
    \item \textbf{Local window tokens}
    \item \textbf{Random tokens}
\end{itemize}

\[
\text{Complexity} = \mathcal{O}(n)
\]

\subsection{Intuition}

\begin{itemize}
    \item Global tokens maintain long-range communication.
    \item Local window captures nearby context.
    \item Random connections ensure theoretical expressivity.
\end{itemize}

\begin{figure}[H]
    \centering
    \includegraphics[width=\textwidth]{BigBird.png}
    \caption{BigBird Sparse Attention Pattern (Global + Local + Random).}
\end{figure}

\subsection{Why BigBird Works}

BigBird was proven to be:
\begin{itemize}
    \item A universal approximator of sequence functions
    \item Turing complete under certain conditions
\end{itemize}

Thus, it preserves theoretical power while reducing computational cost.

\section{FlashAttention: IO-Aware Exact Attention}

FlashAttention does \textbf{NOT} approximate attention.

Instead, it computes exact attention but optimizes memory access.

\subsection{Core Idea}

The real bottleneck is memory bandwidth (HBM), not FLOPs.

FlashAttention:
\begin{enumerate}
    \item Tiles the attention matrix into blocks
    \item Computes each block inside fast SRAM
    \item Uses an online softmax trick to merge results
    \item Avoids materializing the full $n \times n$ matrix
\end{enumerate}

\[
\text{Time Complexity: } \mathcal{O}(n^2)
\]
\[
\text{Memory Usage: Dramatically Reduced}
\]

\begin{figure}[H]
    \centering
    \includegraphics[width=\textwidth]{Flash-Attention.PNG}
    \caption{FlashAttention: Tiled computation inside GPU SRAM.}
\end{figure}

\subsection{Key Insight: Online Softmax}

Instead of computing:

\[
\text{Softmax}(QK^T)
\]

FlashAttention maintains a running max and running sum across tiles, ensuring numerical stability and exact equivalence to full attention.

Thus:
\[
\textbf{Same result as full attention, faster and memory efficient}
\]

\section{BigBird vs FlashAttention}

\begin{table}[H]
\centering
\begin{tabular}{lcc}
\toprule
 & BigBird & FlashAttention \\
\midrule
Attention Type & Sparse & Exact \\
Time Complexity & $\mathcal{O}(n)$ & $\mathcal{O}(n^2)$ \\
Memory Efficiency & High & Very High \\
Long Sequence Scaling & Excellent & Limited by quadratic FLOPs \\
Accuracy vs Full & Approximate & Identical \\
Best Use Case & Very long documents & Large LLM training/inference \\
\bottomrule
\end{tabular}
\caption{Comparison between BigBird and FlashAttention}
\end{table}

\section{When to Use Each?}

\subsection*{Use BigBird When:}
\begin{itemize}
    \item Working with extremely long sequences (8k–100k tokens)
    \item You need linear scaling
    \item Small approximation is acceptable
\end{itemize}

\subsection*{Use FlashAttention When:}
\begin{itemize}
    \item Training large LLMs
    \item You want exact attention
    \item GPU memory bandwidth is bottleneck
\end{itemize}

\section{Real-World Applications}

\textbf{BigBird Used In:}
\begin{itemize}
    \item Long document classification
    \item Question answering over long contexts
    \item Genomics
\end{itemize}

\textbf{FlashAttention Used In:}
\begin{itemize}
    \item GPT-style models
    \item LLaMA variants
    \item Modern production LLM systems
\end{itemize}

\section{Conclusion}

Full attention is powerful but quadratic.

BigBird solves the scaling problem via sparse structure.

FlashAttention solves the memory bottleneck via IO-aware optimization.

They address different dimensions of the same core challenge.

\[
\textbf{BigBird = Algorithmic Efficiency}
\]
\[
\textbf{FlashAttention = Hardware Efficiency}
\]

\end{document}